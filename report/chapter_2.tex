\chapter{Założenia projektowe}

    \section{Opis założeń funkcjonalnych}

    Funkcjonalność kompletnego urządzenia pozwala na bezproblemowy i możliwie najprostszy w realizacji bezdotykowy pomiar temperatury. Urządzenie dokonuje w czasie rzeczywistym pomiaru temperatury danej powierzchni z wykorzystaniem czujnika MLX90614, wyświetlając wynik na wspomnianym wyświetlaczu LCD. 
    
    \vspace{12pt}

    Za pomocą dołączonej 4-przyciskowej klawiatury można:

    \begin{itemize}

        \item zwiększać wartość emisyjności
        \item zmniejszać wartość emisyjności
        \item przywrocić początkową wartość emisyjności wynoszącą 1
        \item zmieniać jednostkę w której wyświetlany jest wynik: stopnie Celsjusza, stopnie Fahrenheita, stopnie Kelvina.

    \end{itemize}

    \section{Opis założeń konstrukcyjnych}
    
    Urządzenie wykonane zostało na płycie ewaluacyjnej, która umożliwia korzystanie z urządzenia, minimalizując ryzyko jakiegokolwiek uszkodzenia urządzenia. Układ działa stabilnie i daje opcję przetransportowania przyrządu.

    \vspace{12pt}

    Wykonany przyrząd pomiarowy składa się z:

    \begin{itemize}

        \item bezdotykowego czujnika temperatury MLX90614
        \item wyświetlacza LCD HD44780 z dołączonym konwerterem I2C
        \item układu sterującego komponentami i przetwarzającymi dane pomiarowe uzyskiwane z czujnika tj. mikrokontrolera Arduino Uno
        \item 4-przyciskowej klawiatury 

    \end{itemize}

    \section{Opis założeń środowiskowych}
    
    

    \section{Opis założeń ekonomicznych}