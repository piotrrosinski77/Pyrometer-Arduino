\chapter{Założenia projektowe}

    \section{Opis założeń funkcjonalnych}

    Funkcjonalność kompletnego urządzenia pozwala na bezproblemowy i możliwie najprostszy w realizacji bezdotykowy pomiar temperatury. Urządzenie dokonuje w czasie rzeczywistym pomiaru temperatury danej powierzchni z wykorzystaniem czujnika MLX90614, wyświetlając wynik na wspomnianym wyświetlaczu LCD. 
    
    \vspace{12pt}

    Za pomocą dołączonej 4-przyciskowej klawiatury można:

    \begin{itemize}

        \item zwiększać wartość emisyjności
        \item zmniejszać wartość emisyjności
        \item przywrocić początkową wartość emisyjności wynoszącą 1
        \item zmieniać jednostkę w której wyświetlany jest wynik wciskając raz za razem przycisk: stopnie Celsjusza, stopnie Fahrenheita, stopnie Kelvina.

    \end{itemize}

    \section{Opis założeń konstrukcyjnych}
    
    Urządzenie wykonane zostało na płycie ewaluacyjnej, która umożliwia korzystanie z urządzenia, minimalizując ryzyko jakiegokolwiek uszkodzenia urządzenia. Układ opierając się konstrukcyjnie na płycie działa stabilnie i daje opcję bezpiecznego przetransportowania przyrządu.

    \vspace{12pt}

    Wykonany w ramach projektu przyrząd pomiarowy składa się z:

    \begin{itemize}

        \item bezdotykowego czujnika temperatury MLX90614, który umożliwia pomiar temperatury obiektu w zakresie -70° do 380°C. Pomiar jest podawany z dokładnością do 0,5°C w zakresie 0-50°C, lub 4°C dla skrajnych wartości zakresu. Natomiast dla temperatury czujnika zakres wynosi od -40°C do 85°C \cite{3}.
        \vspace{12pt}
        \item wyświetlacza LCD HD44780 z dołączonym konwerterem I2C
        \vspace{12pt}
        \item układu sterującego komponentami i przetwarzającymi dane pomiarowe uzyskiwane z czujnika tj. mikrokontrolera Arduino Uno
        \vspace{12pt}
        \item 4-przyciskowej klawiatury 

    \end{itemize}

    \section{Opis założeń środowiskowych}
    
Urządzenie działa w odpowiednio szerokim zakresie temperatur, od -20°C do +70°C, aby umożliwić wykorzystanie w różnych warunkach otoczenia. Zostało ono zaprojektowane tak, aby mogło funkcjonować w umiarkowanej wilgotności tj. do bezpiecznej wartości 60\% przy temperaturze 25°C, tak by zminimalizować ryzyko kondensacji i uszkodzeń komponentów. 

\vspace{12pt}

Urządzenie zasilane jest napięciem 5V (możliwe także zasilanie poprzez port USB 2.0). Urządzenie nie zostało przetestowane pod kątem pracy w trudniejszych warunkach środowiskowych. Użyte materiały są odporne na korozję oraz działanie łagodnych substancji chemicznych, co może mieć znaczenie w przypadku zastosowań przemysłowych bądź laboratoryjnych. 

\vspace{12pt}

Przed wdrożeniem urządzenia do użytku przeprowadzono testy środowiskowe, upewniając się, że spełnia wszystkie założenia dotyczące warunków pracy. Omawiane założenia środowiskowe są kluczowe dla zapewnienia niezawodności i trwałości urządzenia, a także dla jego prawidłowego działania w różnych warunkach otoczenia.

\section{Opis założeń ekonomicznych}
Projekt został opracowany z uwzględnieniem minimalizacji kosztów komponentów z jednoczesnym zachowaniem optymalnej do zastosowań jakości.

\vspace{12pt}

Koszty projektu obejmują: 

\begin{itemize}
    \item Czujnik MLX90614 – koszt jednostkowy w przedziale 50–80 zł w zależności od wybranego dostawcy.
    \vspace{12pt}
    \item Wyświetlacz LCD HD44780 z konwerterem I2C – koszt jednostkowy w przedziale 20–30 zł.
    \vspace{12pt}
    \item Mikrokontroler Arduino Uno – koszt jednostkowy około 100 zł.
    \vspace{12pt}
    \item Klawiatura 4-przyciskowa – koszt jednostkowy w przedziale 3–10 zł.
    \vspace{12pt} 
    \item Laminat miedziowy – koszt jednostkowy w przedziale 20–30 zł.
    \vspace{12pt}
    \item Śruby i elementy łączące - koszt około 5-10 zł.
    \vspace{12pt}
    \item Poliwęglan stanowiący obudowę – koszt jednostkowy około 30-40 zł
\end{itemize}

\vspace{24pt}

Całkowity koszt komponentów to około 220–28 zł, co czyni projekt relatywnie niedrogim i dość przystępnym cenowo w realizacji. Warto jednak zauważyć, że koszty projektu mogą ulec zmianie w zależności od wybranych dostawców i ilości zakupionych komponentów. 

\vspace{12pt}

Urządzenie wykonane jest z części ogólnodostępnych. W jego strukturze nie znajdują się żadne elementy, których pozyskanie mogłoby być problematyczne. Największą część kosztów urządzenia stanowi mikrokontroler Arduino Uno. Jest to najdroższy element, ale jednocześnie kluczowy dla działania całego urządzenia. Istnieją tańsze alternatywy, które pozwolą zredukować koszt urządzenia o około 40\%. Zamiast oryginalnego mikrokontrolera Arduino Uno można zastosować klon, który jest dostępny na rynku w cenie około 30 zł. Warto jednak zauważyć, że jakość klonów może być niższa niż oryginalnego produktu, co może wpłynąć na stabilną pracę i trwałość urządzenia. Istnieją zastosowania, gdzie ze względu na całowitą cenę urządzenia, zastosowanie klonu mikrokontrolera może być uzasadnione. W przypadku omawianego projektu, zastosowanie oryginalnego mikrokontrolera Arduino Uno jest zalecane ze względu na jego niezawodność.