\chapter{Wstęp}
    \section{Wprowadzenie}
    Metrologia optyczna stanowi obecnie jeden z najważniejszych narzędzi pomiarowych w nauce i przemyśle stale zwiększając swoje znaczenie. Bezdotykowy pomiar temperatury rewolucjonizuje precyzję kontroli procesów technologicznych, badań naukowych i diagnostyki medycznej. Szczególną zaletą tych rozwiązań jest możliwość wykonywania pomiarów w warunkach, które dotychczas stanowiły wyzwanie – w przypadku obiektów szybko się poruszających, materiałów o ekstremalnych temperaturach lub gdy klasyczny kontakt pomiarowy mógłby zakłócić naturalne właściwości badanego obiektu i wprowadzić zaburzenie do pomiaru.
    \section{Cel projektu}
    Celem niniejszego projektu jest opracowanie i implementacja pirometru – zaawansowanego urządzenia do bezdotykowego pomiaru temperatury wykorzystującego technologię podczerwieni. Projekt został zrealizowany w oparciu o czujnik MLX90614, który zapewnia odpowiednią precyzję i stabilność pomiarów w założonym zakresie temperatur. Sercem systemu jest popularna płytka mikrokontrolerowa, Arduino UNO, która stanowi centrum sterujące całego urządzenia. Płytka Arduino UNO oparta jest na 8-bitowym mikrokontrolerze ATmega328P, który zapewnia różnorodne funkcje, takie jak 14 cyfrowych pinów wejścia/wyjścia czy 6 analogowych wejść. Dzięki swojej prostocie i wszechstronności, Arduino UNO jest często pierwszym wyborem dla wielu, nieco mniej wymagających obliczeniowo projektów. Kod źródłowy projektu został napisany w języku C/C++, z wykorzystaniem open-sourcowych bibliotek ułatwiających programowanie kluczowych komponentów, w tym wyświetlacza LCD opartego na standardzie HD44780. HD44780 to standardowy kontroler wyświetlaczy LCD. Został opracowany przez firmę Hitachi w latach 80. XX wieku i jest powszechnie stosowany w alfanumerycznych wyświetlaczach dot-matrix \cite{1}.
    \section{Zakres projektu}
    Zakres niniejszego projektu obejmuje kompleksowe opracowanie bezdotykowego systemu pomiarowego temperatury, który łączy optymalne rozwiązania sprzętowe i programowe.