\chapter{Podsumowanie}

Wyniki testów zostały przedstawione w tabeli \ref{tab:porownanie_pomiarow}. Obserwacje wskazują, że:
\begin{itemize}
\item Wyniki pirometru przemysłowego są zbliżone do wyników pirometru testowego, z odchyleniem nieprzekraczającym 4°C w najwyższych temperaturach.
\item Pirometr przemysłowy, jako urządzenie profesjonalne, rejestruje wyższe wartości
temperatur w całym zakresie. Rozbieżności te mogą być związane z niedoskonałością
kalibracji pirometru Arduino, lub ograniczoną rozdzielczością czujnika MLX90614.
\item Termometr stykowy wykazuje wyższe różnice w niższych temperaturach, co
może wynikać z bezwładności termicznej sondy stykowej lub nierównomierności.\item Wraz ze wzrostem temperatury, różnice między wynikami z pirometru Arduino
a pirometru przemysłowego stają się bardziej widoczne. Czujnik MLX90614
odnotowuje wartości niższe niż pirometr przemysłowy w wyższych zakresach
temperatur (powyżej 120°C), co sugeruje możliwość systematycznych błędów
wynikających z niedoskonałości fabrycznej kalibracji. \item Wraz ze wzrostem temperatury, różnice między wynikami z czujnika MLX90614 a pirometru przemysłowego stają się bardziej widoczne. MLX90614
odnotowuje wartości niższe niż pirometr przemysłowy w wyższych zakresach
temperatur (powyżej 120°C), co sugeruje możliwość systematycznych błędów
wynikających z niedoskonałości kalibracji.
%\item Wraz ze wzrostem temperatury, różnice między wynikami z pirometru Arduino
a pirometru przemysłowego stają się bardziej widoczne. Pirometr Arduino
odnotowuje wartości niższe niż pirometr przemysłowy w wyższych zakresach
temperatur (powyżej 120°C), co sugeruje możliwość systematycznych błędów
wynikających z niedoskonałości kalibracji.
\end{itemize}

Podsumowując, pirometr spełnia założenia projektowe i może być stosowany do bezkontaktowego pomiaru temperatury w zakresie od 35°C do 160°C.

%\section{Potencjalne przyczyny błędów}
\begin{itemize}
    \item \textbf{Emisyjność:} Ustawienia emisyjności mają kluczowy wpływ na wyniki pomiarów. Niedokładne dobranie tej wartości dla badanych materiałów może prowadzić do błędów w pomiarach.
    %\item \textbf{Kalibracja:} Pirometr Arduino, jako urządzenie prototypowe, nie posiada profesjonalnej kalibracji fabrycznej, co wpływa na dokładność pomiarów.
    \item \textbf{Czujnik MLX90614:} Czujnik zastosowany w urządzeniu charakteryzuje się ograniczoną dokładnością w wyższych zakresach temperatur, co mogło wpłynąć na odchylenia w pomiarach.
    \item \textbf{Wpływ środowiska:} Czynniki takie jak wilgotność, temperatura otoczenia czy odbicia promieniowania podczerwonego mogą wprowadzać dodatkowe błędy.
\end{itemize}

%\addcontentsline{toc}{chapter}{Bibliografia}

\cleardoublepage
\phantomsection
\addcontentsline{toc}{chapter}{Bibliografia}
\begin{thebibliography}{999}
\begin{spacing}{1}
\end{spacing}
\end{thebibliography}

\chapter*{Dodatki}

\section*{Pełny listing kodu źródłowego}

\begin{lstlisting}[language=C++]
    // Source code implementing the MLX90614 sensor with an LCD display and keypad as a pyrometer using the Arduino platform

    // Authors :
    // Patryk Niczke
    // Piotr Rosinski
    // Przemyslaw Lis

    #include <Arduino.h>
    #include <Adafruit_MLX90614.h>
    #include <LiquidCrystal_I2C.h>
    #include <math.h>
    
    // Define buttons
    #define BTN1 2
    #define BTN2 3
    #define BTN3 4
    #define BTN4 5
    
    // Set I2C address for the LCD (change if needed, e.g., 0x27)
    LiquidCrystal_I2C lcd(0x27, 16, 2); // 16 chars, 2 lines
    // Custom epsilon character
    byte epsilon[8] = { B00000, B00000, B01110, B10000, B11110, B10000, B01110, B00000 };

    // Initialize MLX90614 sensor
    Adafruit_MLX90614 mlx = Adafruit_MLX90614();
    float ems = 1.0;
    // Default emissivity
    int tempScale = 0; // 0 - Celsius, 1 - Fahrenheit, 2 - Kelvin
    
    float correctTemperature(float measuredTemp, float ambientTemp, float emissivity) {
    // Convert measured temperature from Celsius to Kelvin
    float measuredTempK = measuredTemp + 273.15;
    // Convert ambient temperature from Celsius to Kelvin
    float ambientTempK = ambientTemp + 273.15;
    // Calculate true temperature in Kelvin
    float trueTempK = pow((pow(measuredTempK, 4) - (1 - emissivity) * pow(ambientTempK, 4)) / emissivity, 0.25);
    // Convert true temperature from Kelvin to Celsius
    return trueTempK - 273.15;
}

void setup() {
    delay(200);
    // Set button pins as input with pull-up resistors
    pinMode(BTN1, INPUT_PULLUP);
    pinMode(BTN2, INPUT_PULLUP);
    pinMode(BTN3, INPUT_PULLUP);
    pinMode(BTN4, INPUT_PULLUP);
    
    // Initialize serial communication
    Serial.begin(9600);
    // Initialize LCD display
    lcd.init();
    lcd.clear();
    lcd.backlight();
    lcd.setCursor(0, 0);
    lcd.createChar(0, epsilon);
    
    // Check if emissivity value is valid
    if (isnan(ems)) {
        ems = 1.0;
    }
    // Initialize MLX90614 sensor
    if (!mlx.begin()) {
        lcd.setCursor(0, 1);
        lcd.print("MLX error!");
        Serial.print("MLX error!\n");
        while (1);
    }
}

void loop() {
    // Read button values
    int BTN1V = digitalRead(BTN1);
    int BTN2V = digitalRead(BTN2);
    int BTN3V = digitalRead(BTN3);
    int BTN4V = digitalRead(BTN4);
    
    // Increase emissivity
    if (!BTN1V && ems < 1.0) {
        Serial.println("Increased emissivity");
        ems += 0.01;
        if (ems > 1.0) ems = 1.0;
    }
    // Decrease emissivity
    if (!BTN2V && ems > 0.0) {
        Serial.println("Decreased emissivity");
        ems -= 0.01;
        if (ems < 0.0) ems = 0.0;
    }
    // Reset emissivity
    if (!BTN3V) {
        Serial.println("Emissivity reset");
        ems = 1.0;
    }
    // Switch temperature scale
    if (!BTN4V) {
        tempScale = (tempScale + 1) % 3;
        Serial.println(tempScale == 0 ? "Switched to Celsius" : (tempScale == 1 ? "Switched to Fahrenheit" : "Switched to Kelvin"));
        delay(300);
    } 
    // Read object and ambient temperatures
    float ObjTemp = mlx.readObjectTempC();
    float AmbientTemp = mlx.readAmbientTempC();
    // Correct object temperature
    float correctedTemp = correctTemperature(ObjTemp, AmbientTemp, ems);
    
    // Check if temperature is valid
    if (isnan(correctedTemp)) {
        Serial.println("Read error: Temperature NaN");
        correctedTemp = 0.0;
    }
    
    // Display temperature on LCD
    float displayTemp = correctedTemp;
    char scaleLabel = 'C';
    bool showDegreeSymbol = true;
    
    // Convert temperature to selected scale
    if (tempScale == 1) {
        displayTemp = correctedTemp * 9.0 / 5.0 + 32.0;
        scaleLabel = 'F';
    } else if (tempScale == 2) {
        displayTemp = correctedTemp + 273.15;
        scaleLabel = 'K';
        showDegreeSymbol = false;
    }
    
    // Display temperature on serial monitor
    Serial.print("Temperature: ");
    Serial.print(displayTemp);
    Serial.print(" ");
    Serial.print(scaleLabel);
    Serial.print("\nEnvironment Temperature: ");
    Serial.print(AmbientTemp);
    Serial.print("\nEmissivity: ");
    Serial.println(ems);
    
    // Display temperature on LCD
    lcd.setCursor(0, 0);
    lcd.print("Temp: ");
    lcd.print(displayTemp);
    if (showDegreeSymbol) {
        lcd.print((char)223);
    }
    lcd.print(scaleLabel);
    // Display emissivity on LCD
    lcd.setCursor(0, 1);
    lcd.write(byte(0));
    lcd.print(": ");
    lcd.print(ems);
    delay(500);
    }
    \end{lstlisting}

    \chapter*{Program ćwiczenia dla studentów}

    %\section*{Program ćwiczenia dla studentów}

\subsection*{BADANIE TEMPERATURY POWIERZCHNI CIEPŁOWYTWARZAJĄCYCH MATERIAŁÓW}

\subsubsection*{Cel ćwiczenia}
Celem ćwiczenia jest pomiar temperatury powierzchni różnych materiałów ciepłowytwarzających za pomocą pirometru oraz analiza wpływu rodzaju materiału i koloru powierzchni na dokładność pomiaru.

\subsubsection*{Potrzebne materiały}
Pirometr, materiały ciepłowytwarzające o różnych kolorach i właściwościach emisji (np. metal, drewno, plastik), źródło ciepła, statyw do pirometru, taśmy lub oznaczniki do identyfikacji powierzchni.

\subsubsection*{Przebieg ćwiczenia}
\begin{enumerate}
    \item Przygotowanie próbek materiałów o różnych właściwościach emisji i kolorach:
    \begin{itemize}
        \item Ustaw próbki materiałów w bezpiecznej odległości od źródła ciepła.
        \item Oznacz próbki w celu ich łatwej identyfikacji.
    \end{itemize}
    \item Pomiar temperatury za pomocą pirometru:
    \begin{itemize}
        \item Skieruj pirometr na powierzchnię każdej próbki z jednakowej odległości.
        \item Zanotuj odczyt temperatury z wyświetlacza pirometru dla każdej próbki.
    \end{itemize}
    \item Analiza wyników:
    \begin{itemize}
        \item Porównaj odczyty temperatury dla próbek wykonanych z różnych materiałów.
        \item Zastanów się, jak kolor i właściwości emisji powierzchni wpływają na dokładność pomiaru.
    \end{itemize}
\end{enumerate}

\subsubsection*{Wnioski}
Na podstawie wyników pomiarów i analizy wyciągnij wnioski dotyczące wpływu rodzaju materiału i koloru powierzchni na dokładność pomiaru temperatury pirometrem. Oceń również potencjalne źródła błędów w przeprowadzonym ćwiczeniu.

\subsubsection*{Dodatkowo}
Zaprojektuj eksperyment, w którym porównasz odczyty temperatury dla tych samych materiałów, ale o różnym wykończeniu powierzchni (np. matowym i błyszczącym). Omów wyniki w kontekście właściwości emisji powierzchni badanych próbek.