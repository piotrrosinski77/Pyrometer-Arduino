\cleardoublepage
\phantomsection
\addcontentsline{toc}{chapter}{Bibliografia}
\begin{thebibliography}{999}
\begin{spacing}{1}

    %\bibitem{1} Sanchez, Julio; Canton, Maria P. (2007) - „Microcontroller Programming: the Microchip PIC. CRC Press.”

    %\bibitem{2} ElektronikaB2B - „Arduino - jak wybrać i kupić?”
    
    %\url{https://elektronikab2b.pl/technika/50150-arduino-jak-wybrac-i-kupic}

    %\bibitem{3} Korneliusz Jarzębski - „Pirometr z czujnikiem MLX90614ESF-BAA”

    %\url{https://www.jarzebski.pl/arduino/czujniki-i-sensory/pirometr-z-czujnikiem-mlx90614.html}

    %\bibitem{4} Arduino - „Arduino Uno Rev3”

    %\url{https://docs.arduino.cc/hardware/uno-rev3/#features}

    %\bibitem{5} ElektroWeb - „Pirometr termometr bezdotykowy MLX90614 GY-906”
    
    %\url{https://www.elektroweb.pl/pl/czujniki-temperatury/273-pirometr-termometr-bezdotykowy-mlx90614-gy-906.html}

    \bibitem{1} „Metody i urządzenia do pomiaru temperatury i ciśnienia” - Dr inż. Andrzej Kłabut

    \url{https://slideplayer.pl/slide/5307334/17/images/4/Metody+pomiaru+temperatury.jpg}

    \bibitem{2} „Pirometr - poradnik użytkowania” - MERA-SP.PL

    \url{https://mera-sp.pl/blog/rozwiazania/pirometry-poradnik-uzytkowania}

    \bibitem{3} „Rozkład energii w widmie promieniowania ciała doskonale czarnego w różnych temperaturach”

    \url{https://weblab.deusto.es/olarex/cd/kaernten/BBR_PLnew_27.09.2013/rozkad_energii_w_widmie_promieniowania_ciaa_doskonale_czarnego_w_rnych_temperaturach.html}

    \bibitem{4} Korneliusz Jarzębski - „Pirometr z czujnikiem MLX90614ESF-BAA”

    \url{https://www.jarzebski.pl/arduino/czujniki-i-sensory/pirometr-z-czujnikiem-mlx90614.html}

    \bibitem{5} Arduino - „Arduino Uno Rev3”

    \url{https://store.arduino.cc/products/arduino-uno-rev3/}

     \bibitem{6} ElektroWeb - „Pirometr termometr bezdotykowy MLX90614 GY-906”
    
    \url{https://www.elektroweb.pl/pl/czujniki-temperatury/273-pirometr-termometr-bezdotykowy-mlx90614-gy-906.html}

\end{spacing}
\end{thebibliography}