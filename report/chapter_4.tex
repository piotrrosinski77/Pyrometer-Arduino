\chapter{Opis części programowej}

Oprogramowanie pirometru zostało napisane w języku C++ z wykorzystaniem środowiska Arduino IDE. Program odpowiada za odczyt temperatury z czujnika MLX90614, wyświetlanie wyników na ekranie LCD 16x2 oraz obsługę 4-przyciskowej klawiatury do regulacji emisyjności/zmiany jednostki. Kod został zoptymalizowany pod kątem intuicyjnej obsługi i wysokiej dokładności pomiarów. Podstawowy algorytm działania przedstawiono na schemacie blokowym \ref{fig:blok}.

\vspace{24pt}

\begin{figure}[h!]
    \centering
    \includegraphics[width=1.1\textwidth]{images/program.png}
    \caption{Schemat blokowy skonstruowanego urządzenia} 
    \label{fig:blok}
\end{figure}

%\vspace{12pt}

\section{Biblioteki i wstępna konfiguracja}

Na początku kodu są załączone biblioteki:

\texttt{\#include <Arduino.h>} – standardowa biblioteka Arduino.

\texttt{\#include <Adafruit\_MLX90614.h>} – obsługa czujnika MLX90614.

\texttt{\#include <LiquidCrystal\_I2C.h>} – sterowanie LCD przez I2C.

\texttt{\#include <math.h>} – funkcje matematyczne (\texttt{pow()}, \texttt{sqrt()}).

\section*{Definiowanie przycisków}
Numery pinów dla czterech przycisków:
\begin{lstlisting}[language=C++]
#define BTN1 2
#define BTN2 3
#define BTN3 4
#define BTN4 5
\end{lstlisting}
Przyciski podłączone do pinów cyfrowych 2–5.

\section*{Konfiguracja wyświetlacza LCD}
\begin{lstlisting}[language=C++]
LiquidCrystalI2C lcd(0x27, 16, 2);
\end{lstlisting}
Adres I2C: \texttt{0x27}, wyświetlacz: 16 kolumn, 2 wiersze.

\section*{Definiowanie znaku \(\epsilon\)}
\begin{lstlisting}[language=C++]
byte epsilon[8] = { 
    B00000, B00000, B01110, B10000, 
    B11110, B10000, B01110, B00000 
};
\end{lstlisting}
Znak \(\epsilon\) (8x5 pikseli) do wyświetlenia na LCD.

\section*{Inicjalizacja czujnika MLX90614}
\begin{lstlisting}[language=C++]
Adafruit_MLX90614 mlx = Adafruit_MLX90614();
float ems = 1.0;
int tempScale = 0;
\end{lstlisting}
\begin{itemize}
    \item \texttt{mlx} – obiekt czujnika MLX90614.
    \item \texttt{ems = 1.0} – emisyjność (domyślna wartość).
    \item \texttt{tempScale} – skala temperatury: 0 (°C), 1 (°F), 2 (K).
\end{itemize}

\section{Połączenie z klawiaturą}

\section*{Odczyt stanów przycisków}
\begin{lstlisting}[language=C++]
int BTN1V = digitalRead(BTN1);
int BTN2V = digitalRead(BTN2);
int BTN3V = digitalRead(BTN3);
int BTN4V = digitalRead(BTN4);
\end{lstlisting}
Funkcja \texttt{digitalRead()} odczytuje stan każdego z przycisków (BTN1 do BTN4). Zwraca 0 (przycisk wciśnięty) lub 1 (przycisk nie wciśnięty). Zmienna \texttt{BTN1V}, \texttt{BTN2V}, \texttt{BTN3V} i \texttt{BTN4V} przechowują odpowiednio stany przycisków 1, 2, 3 i 4.

\section*{Zwiększenie emisyjności (emissivity)}
\begin{lstlisting}[language=C++]
if (!BTN1V && ems < 1.0) {
    Serial.println("Increased emissivity");
    ems += 0.01;
    if (ems > 1.0) ems = 1.0;
}
\end{lstlisting}
Jeśli przycisk \texttt{BTN1} jest wciśnięty i emisyjność (\texttt{ems}) jest mniejsza niż 1.0, program:
\begin{itemize}
    \item Wyświetla w monitorze szeregowym komunikat \texttt{"Increased emissivity"}.
    \item Zwiększa wartość emisyjności o 0.01.
    \item Upewnia się, że emisyjność nie przekroczy 1.0 (maksymalna dopuszczalna wartość).
\end{itemize}

\section*{Zmniejszenie emisyjności}
\begin{lstlisting}[language=C++]
if (!BTN2V && ems > 0.0) {
    Serial.println("Decreased emissivity");
    ems -= 0.01;
    if (ems < 0.0) ems = 0.0;
}
\end{lstlisting}
Jeśli przycisk \texttt{BTN2} jest wciśnięty i emisyjność (\texttt{ems}) jest większa niż 0.0, program:
\begin{itemize}
    \item Wyświetla w monitorze szeregowym komunikat \texttt{"Decreased emissivity"}.
    \item Zmniejsza wartość emisyjności o 0.01.
    \item Upewnia się, że emisyjność nie spadnie poniżej 0.0 (minimalna dopuszczalna wartość).
\end{itemize}

\section*{Resetowanie emisyjności}
\begin{lstlisting}[language=C++]
if (!BTN3V) {
    Serial.println("Emissivity reset");
    ems = 1.0;
}
\end{lstlisting}
Jeśli przycisk \texttt{BTN3} jest wciśnięty, program:
\begin{itemize}
    \item Wyświetla w monitorze szeregowym komunikat \texttt{"Emissivity reset"}.
    \item Ustawia wartość emisyjności na 1.0.
\end{itemize}

\section*{Zmiana jednostki}
\begin{lstlisting}[language=C++]
if (!BTN4V) {
    tempScale = (tempScale + 1) % 3;
    Serial.println(tempScale == 0 ? "Switched to Celsius" : 
                    (tempScale == 1 ? "Switched to Fahrenheit" : 
                    "Switched to Kelvin"));
    delay(300);
}
\end{lstlisting}
Jeśli przycisk \texttt{BTN4} jest wciśnięty, program:
\begin{itemize}
    \item Zmienia zmienną \texttt{tempScale}, aby cyklicznie przechodziła przez trzy wartości (Celsius, Fahrenheit, Kelvin). Zmienna \texttt{tempScale} jest inkrementowana, a następnie obliczany jest reszta z dzielenia przez 3, co daje efekt cyklicznego przełączania między 0, 1, 2.
    \item Wyświetla odpowiedni komunikat w monitorze szeregowym:
    \begin{itemize}
        \item \texttt{"Switched to Celsius"} – jeśli \texttt{tempScale} wynosi 0.
        \item \texttt{"Switched to Fahrenheit"} – jeśli \texttt{tempScale} wynosi 1.
        \item \texttt{"Switched to Kelvin"} – jeśli \texttt{tempScale} wynosi 2.
    \end{itemize}
    \item Funkcja \texttt{delay(300)} wprowadza opóźnienie 300 ms, aby przycisk nie wywołał wielu zmian w krótkim czasie.
\end{itemize}


%\section{Połączenie z klawiaturą}



\section{Połączenie z czujnikiem temperatury MLX90614}

\section*{Odczyt temperatury}

\begin{lstlisting}[language=C++]
float ObjTemp = mlx.readObjectTempC();
float AmbientTemp = mlx.readAmbientTempC();
float correctedTemp = correctTemperature(ObjTemp, AmbientTemp, ems);
\end{lstlisting}

W tych liniach kodu odczytywane są dwie temperatury z czujnika MLX90614:
\begin{itemize}
    \item \texttt{ObjTemp}: temperatura obiektu, którą czujnik odczytuje w stopniach Celsjusza.
    \item \texttt{AmbientTemp}: temperatura otoczenia, również odczytywana w stopniach Celsjusza.
\end{itemize}

Następnie obliczana jest \texttt{correctedTemp}, czyli poprawiona temperatura obiektu na podstawie zmierzonej temperatury obiektu, temperatury otoczenia i wartości emisyjności, wykorzystując funkcję \texttt{correctTemperature()}.

\begin{lstlisting}[language=C++]
if (isnan(correctedTemp)) {
    Serial.println("Read error: Temperature NaN");
    correctedTemp = 0.0;
}
\end{lstlisting}

Jeśli funkcja \texttt{correctTemperature()} zwróci wartość "Not a Number" (NaN), co oznacza, że wystąpił błąd w obliczeniach, program wyświetla komunikat o błędzie na monitorze szeregowym, a zmienna \texttt{correctedTemp} jest ustawiana na 0.0.


\section{Połączenie z wyświetlaczem LCD HD44780}

%W tym fragmencie kodu tworzony jest obiekt \texttt{lcd}, który reprezentuje wyświetlacz LCD podłączony do mikrokontrolera przez interfejs I2C. Używany jest adres \texttt{0x27}, który jest najczęściej spotykanym adresem dla wyświetlaczy LCD I2C. Liczba kolumn to 16, a liczba wierszy to 2, co oznacza wyświetlacz o wymiarach 16x2.

\section*{Inicjalizacja wyświetlacza LCD}

\begin{lstlisting}[language=C++]
lcd.init();
lcd.clear();
lcd.backlight();
lcd.setCursor(0, 0);
lcd.createChar(0, epsilon);
\end{lstlisting}

Po zainicjowaniu wyświetlacza za pomocą \texttt{lcd.init()}, wyświetlacz zostaje wyczyszczony (\texttt{lcd.clear()}), a podświetlenie zostaje włączone (\texttt{lcd.backlight()}). Kursor jest ustawiany na początek pierwszego wiersza (\texttt{lcd.setCursor(0, 0)}), a funkcja \texttt{lcd.createChar()} pozwala na zdefiniowanie niestandardowego znaku, np. symbolu epsilon, który może być użyty w wyświetlaczu.

\begin{lstlisting}[language=C++]
lcd.setCursor(0, 0);
lcd.print("Temp: ");
lcd.print(displayTemp);
lcd.print(scaleLabel);
\end{lstlisting}

W tym fragmencie kodu tekst jest wyświetlany na LCD. Funkcja \texttt{lcd.setCursor(0, 0)} ustawia kursor na początku pierwszego wiersza, a następnie wyświetlany jest tekst "Temp: ", wartość temperatury oraz jednostka skali temperatury (°C, °F, K).

\begin{lstlisting}[language=C++]
lcd.setCursor(0, 1);
lcd.write(byte(0));
lcd.print(" : ");
lcd.print(ems);
\end{lstlisting}

Drugi wiersz wyświetlacza pokazuje wartość emisyjności. Funkcja \texttt{lcd.write(byte(0))} wyświetla wcześniej zdefiniowany niestandardowy znak (np. epsilon), a następnie na ekranie pojawia się wartość emisyjności (\texttt{ems}).

%\section{Połączenie z czujnikiem temperatury MLX90614}
%\section{Połączenie z wyświetlaczem LCD HD44780}
%\section{Synchroniczna współpraca LCD i czujnika temperatury z wykorzystaniem mikrokontrolera Arduino}
%W celu synchronicznej współpracy wyświetlacza LCD i czujnika temperatury z mikrokontrolerem Arduino, został napisany program, który odczytuje temperaturę z czujnika i wyświetla ją na monitorze szeregowym i wyświetlaczu LCD. Program został napisany w języku C/C++ z wykorzystaniem bibliotek Wire i LiquidCrystal I2C. % W celu komunikacji z czujnikiem została wykorzystana biblioteka Wire.h.% W celu sprawdzenia poprawności połączenia z czujnikiem został napisany program, który odczytuje temperaturę z czujnika i wyświetla ją na monitorze szeregowym i wyświetlaczu LCD.
%\section{Wykonanie płyty ewaluacyjnej oraz konstrukcja gotowego urządzenia naukowo-badawczego}

\chapter{Uruchomienie, kalibracja}

Po podłączeniu urządzenia do zasilania za pomocą kabla USB, włączy się ono automatycznie. Czujnik nie wymaga przeprowadzenia kalibracji. W przypadku braku odczytów na wyświetlaczu, należy dostosować kontrast przy pomocy potencjometru umiejscowionego po lewej stronie ekranu. Wszelkie problemy z komunikacją z czujnikiem lub wyświetlaczem będą przekazywane przez interfejs szeregowy, którego transmisję można monitorować za pomocą programu Arduino IDE.

%\chapter{Opis wzorów fizycznych}
% Poniżej przedstawiono zestaw wzorów opisujących wymianę ciepła przez promieniowanie oraz związane z nimi parametry fizyczne:
% \begin{itemize}
%     \item \(\sigma\) – stała Stefana-Boltzmanna, określająca intensywność promieniowania ciała doskonale czarnego,
%     \item \(\epsilon\) – współczynnik emisyjności (od 0 do 1), opisujący zdolność ciała do emitowania promieniowania w stosunku do ciała doskonale czarnego,
%     \item \(S\) – powierzchnia ciała emitującego promieniowanie,
%     \item \(T_{\text{env}}\) – temperatura otoczenia w stopniach Celcjusza (\(C\)),
%     \item \(T_{\text{meas}}\) – zmierzona temperatura obiektu stopniach Celcjusza (\(C\)),
%     \item \(T_{\text{real}}\) – rzeczywista temperatura obiektu stopniach Celcjusza (\(C\)).
% \end{itemize}
% \newpage
% %\section{Wzory}
% 1. Moc promieniowania cieplnego emitowanego przez ciało:
% \[
% P = \sigma \cdot \epsilon \cdot S \cdot \left( T_{\text{env}}^4 - T^4 \right)
% \]
% 2. Równanie równowagi cieplnej opisujące emisję promieniowania:
% \[
% \epsilon \cdot T_{\text{env}}^4 - \epsilon \cdot T_{\text{real}}^4 = T_{\text{env}}^4 - T_{\text{meas}}^4
% \]
% 3. Współczynnik emisyjności obliczony na podstawie temperatur:
% \[
% \epsilon = \frac{T_{\text{env}}^4 - T_{\text{meas}}^4}{T_{\text{env}}^4 - T_{\text{real}}^4}
% \]
% Na podstawie dwóch różnych temperatur wyznaczono emisyjność badanego obiektu. 
% \vspace{12pt}
% Wzory zostały zastosowane do obliczenia wartości emisyjności:
% \[
% \epsilon = \frac{30.15^4 - 69.31^4}{30.15^4 - 70.1^4} = 0.9541
% \]
% Dokładny wynik obliczenia przed zaokrągleniem wynosi: 0.9540835302172766, po zaokrągleniu do czterech miejsc po przecinku wynik to: 0.9541.