\chapter{Opis części programowej}
%\section{Połączenie z czujnikiem temperatury MLX90614}

 

%\section{Połączenie z wyświetlaczem LCD HD44780}



%\section{Synchroniczna współpraca LCD i czujnika temperatury z wykorzystaniem mikrokontrolera Arduino}

%W celu synchronicznej współpracy wyświetlacza LCD i czujnika temperatury z mikrokontrolerem Arduino, został napisany program, który odczytuje temperaturę z czujnika i wyświetla ją na monitorze szeregowym i wyświetlaczu LCD. Program został napisany w języku C/C++ z wykorzystaniem bibliotek Wire i LiquidCrystal I2C. % W celu komunikacji z czujnikiem została wykorzystana biblioteka Wire.h.% W celu sprawdzenia poprawności połączenia z czujnikiem został napisany program, który odczytuje temperaturę z czujnika i wyświetla ją na monitorze szeregowym i wyświetlaczu LCD.

%\section{Wykonanie płyty ewaluacyjnej oraz konstrukcja gotowego urządzenia naukowo-badawczego}

\chapter{Uruchomienie, kalibracja}

%\chapter{Opis wzorów fizycznych}

% Poniżej przedstawiono zestaw wzorów opisujących wymianę ciepła przez promieniowanie oraz związane z nimi parametry fizyczne:

% \begin{itemize}
%     \item \(\sigma\) – stała Stefana-Boltzmanna, określająca intensywność promieniowania ciała doskonale czarnego,
%     \item \(\epsilon\) – współczynnik emisyjności (od 0 do 1), opisujący zdolność ciała do emitowania promieniowania w stosunku do ciała doskonale czarnego,
%     \item \(S\) – powierzchnia ciała emitującego promieniowanie,
%     \item \(T_{\text{env}}\) – temperatura otoczenia w stopniach Celcjusza (\(C\)),
%     \item \(T_{\text{meas}}\) – zmierzona temperatura obiektu stopniach Celcjusza (\(C\)),
%     \item \(T_{\text{real}}\) – rzeczywista temperatura obiektu stopniach Celcjusza (\(C\)).
% \end{itemize}

% \newpage

% %\section{Wzory}
% 1. Moc promieniowania cieplnego emitowanego przez ciało:
% \[
% P = \sigma \cdot \epsilon \cdot S \cdot \left( T_{\text{env}}^4 - T^4 \right)
% \]

% 2. Równanie równowagi cieplnej opisujące emisję promieniowania:
% \[
% \epsilon \cdot T_{\text{env}}^4 - \epsilon \cdot T_{\text{real}}^4 = T_{\text{env}}^4 - T_{\text{meas}}^4
% \]

% 3. Współczynnik emisyjności obliczony na podstawie temperatur:
% \[
% \epsilon = \frac{T_{\text{env}}^4 - T_{\text{meas}}^4}{T_{\text{env}}^4 - T_{\text{real}}^4}
% \]

% Na podstawie dwóch różnych temperatur wyznaczono emisyjność badanego obiektu. 

% \vspace{12pt}

% Wzory zostały zastosowane do obliczenia wartości emisyjności:

% \[
% \epsilon = \frac{30.15^4 - 69.31^4}{30.15^4 - 70.1^4} = 0.9541
% \]
% Dokładny wynik obliczenia przed zaokrągleniem wynosi: 0.9540835302172766, po zaokrągleniu do czterech miejsc po przecinku wynik to: 0.9541.